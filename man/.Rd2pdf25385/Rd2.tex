\documentclass[a4paper]{book}
\usepackage[times,inconsolata,hyper]{Rd}
\usepackage{makeidx}
\usepackage[latin1]{inputenc} % @SET ENCODING@
% \usepackage{graphicx} % @USE GRAPHICX@
\makeindex{}
\begin{document}
\chapter*{}
\begin{center}
{\textbf{\huge \R{} documentation}} \par\bigskip{{\Large of \file{./add\_data.Rd} etc.}}
\par\bigskip{\large \today}
\end{center}
\inputencoding{utf8}
\HeaderA{add\_data}{Add to probe-node matrix at a new node}{add.Rul.data}
%
\begin{Description}\relax
At a given node, this function adds data to the probe-node matrix at node 
n\_new. Make sure that the sizes of data and the probe-node matrix are 
compatible before calling this function.
\end{Description}
%
\begin{Usage}
\begin{verbatim}
add_data(probe_node_matrix, n_new., data)
\end{verbatim}
\end{Usage}
%
\begin{Arguments}
\begin{ldescription}
\item[\code{probe\_node\_matrix}] matrix storing the methylation status across all
probes and nodes on a tree

\item[\code{data}] column matrix of methylation status for a node across many 
probes

\item[\code{n\_new}] node number
\end{ldescription}
\end{Arguments}
%
\begin{Value}
probe-node matrix with data stored at node n\_new
\end{Value}
%
\begin{Examples}
\begin{ExampleCode}
tree <- add_data(probe_node_matrix, 12, probe_node_matrix['6'])
... some visualization 
\end{ExampleCode}
\end{Examples}
\inputencoding{utf8}
\HeaderA{betas\_to\_consensus\_vector}{Converts betas matrix to consensus vectors of probes for each group  VALIDATED}{betas.Rul.to.Rul.consensus.Rul.vector}
%
\begin{Description}\relax
This function creates a consensus vector for each group using the betas 
matrix
\end{Description}
%
\begin{Usage}
\begin{verbatim}
betas_to_consensus_vector(
  reference_file,
  series_matrix_file,
  betas,
  group_names,
  remove_names
)
\end{verbatim}
\end{Usage}
%
\begin{Arguments}
\begin{ldescription}
\item[\code{reference\_file}] directory location of where given IDAT file is located

\item[\code{series\_matrix\_file}] directory location of where given series matrix 
file is located

\item[\code{betas}] matrix (rows are cell names, columns are probe names)

\item[\code{group\_names}] list of cell names in betas

\item[\code{remove\_names}] list of cell names in betas
\end{ldescription}
\end{Arguments}
%
\begin{Value}
consensus vector data frame for each cell in group names
\end{Value}
%
\begin{Examples}
\begin{ExampleCode}
consensus_vector <- betas_to_consensus_vector('~/references/
EPIC.hg38.manifest.rds', 'GSE110554/samples_GEOLoadSeriesMatrix.rds', 
betas, c('NK', "Monocytes", ...), c('A.mix', 'B.mix', ...))
... some visualization 
\end{ExampleCode}
\end{Examples}
\inputencoding{utf8}
\HeaderA{create\_consensus\_vector}{Create consensus vector of probes for each group VALIDATED}{create.Rul.consensus.Rul.vector}
%
\begin{Description}\relax
This function creates consensus vector for each group using majority ruling:
across all probes, whichever category (methylated/unmethylated) is 
represented in more than 2/3's of the sample will saved. Otherwise, -1 for 
ambiguous is saved. A probabilistic model will supplement this in the future.
\end{Description}
%
\begin{Usage}
\begin{verbatim}
create_consensus_vector(betas, group_names)
\end{verbatim}
\end{Usage}
%
\begin{Arguments}
\begin{ldescription}
\item[\code{betas}] matrix (rows are cell names, columns are probe names)

\item[\code{group\_names}] list of cell names in betas
\end{ldescription}
\end{Arguments}
%
\begin{Value}
consensus vector for each cell in group names
\end{Value}
%
\begin{Examples}
\begin{ExampleCode}
consensus_vector <- create_consensus_vector(betas, c('NK', "Monocytes", 
...))
... some visualization 
\end{ExampleCode}
\end{Examples}
\inputencoding{utf8}
\HeaderA{create\_node\_matrix}{Node matrix used for tree walking VALIDATED}{create.Rul.node.Rul.matrix}
%
\begin{Description}\relax
Returns a matrix which tracks where recursive functions are in a tree 
structure. This is past to most recursive functions in the pipeline.
\end{Description}
%
\begin{Usage}
\begin{verbatim}
create_node_matrix(tree)
\end{verbatim}
\end{Usage}
%
\begin{Arguments}
\begin{ldescription}
\item[\code{tree}] ape::phylo tree
\end{ldescription}
\end{Arguments}
%
\begin{Value}
node matrix with one column for ancestors and another for 
descendants
\end{Value}
%
\begin{Examples}
\begin{ExampleCode}
node_matrix <- create_node_matrix(tree)
... some visualization 
\end{ExampleCode}
\end{Examples}
\inputencoding{utf8}
\HeaderA{cut\_tree}{Recursively walk through tree and remove descendants.}{cut.Rul.tree}
%
\begin{Description}\relax
This function will recursively walk through a given tree and remove all of 
node n's descendants, making node n a new leaf.
\end{Description}
%
\begin{Usage}
\begin{verbatim}
cut_tree(node, root, node_matrix, i)
\end{verbatim}
\end{Usage}
%
\begin{Arguments}
\begin{ldescription}
\item[\code{node}] node number

\item[\code{root}] root of non-proper subtree

\item[\code{node\_matrix}] matrix obtained from create\_node\_matrix function

\item[\code{i}] row in node matrix, start at 1
\end{ldescription}
\end{Arguments}
%
\begin{Value}
None
\end{Value}
%
\begin{Examples}
\begin{ExampleCode}
cut_tree(node, get_root(tree), node_matrix, 1)
... some visualization 
\end{ExampleCode}
\end{Examples}
\inputencoding{utf8}
\HeaderA{fit\_betas\_to\_distribution}{Categorize betas based on bimodal distribution VALIDATED}{fit.Rul.betas.Rul.to.Rul.distribution}
%
\begin{Description}\relax
This function uses a given bimodal distribution generated within cells
in order to categorize betas as either methylated (1), unmethylated (0), or 
ambiguous (-1).
\end{Description}
%
\begin{Usage}
\begin{verbatim}
fit_betas_to_distribution(betas, bi)
\end{verbatim}
\end{Usage}
%
\begin{Arguments}
\begin{ldescription}
\item[\code{betas}] matrix (rows are cell names, columns are probe names)

\item[\code{bi}] matrix specifying mu0, mu1, sigma0, sigma1 across all cell types
\end{ldescription}
\end{Arguments}
%
\begin{Value}
betas categorized based on bimodal distribution
\end{Value}
%
\begin{Examples}
\begin{ExampleCode}
betas <- fit_betas_to_distribution(betas, bi)
... some visualization 
\end{ExampleCode}
\end{Examples}
\inputencoding{utf8}
\HeaderA{generate\_bimodal\_distribution}{Generate bimodal distribution VALIDATED}{generate.Rul.bimodal.Rul.distribution}
%
\begin{Description}\relax
This function generates a bimodal distribution on the betas matrix data 
within cells
\end{Description}
%
\begin{Usage}
\begin{verbatim}
generate_bimodal_distribution(betas)
\end{verbatim}
\end{Usage}
%
\begin{Arguments}
\begin{ldescription}
\item[\code{betas}] matrix (rows are cell names, columns are probe names)
\end{ldescription}
\end{Arguments}
%
\begin{Value}
matrix specifying mu0, mu1, sigma0, sigma1 across all cell types
\end{Value}
%
\begin{Examples}
\begin{ExampleCode}
bi <- generate_bimodal_distribution(betas)
... some visualization 
\end{ExampleCode}
\end{Examples}
\inputencoding{utf8}
\HeaderA{generate\_branch\_changes}{Relevant probes for gain, loss, and change in methylation status VALIDATED}{generate.Rul.branch.Rul.changes}
%
\begin{Description}\relax
This function returns relevant probe lists for each node that exhibit a 
gain or loss in methylation status. It also returns three sets of branch 
labels: number of methylation changes, number of methylation gains, and 
number of methylation losses.
\end{Description}
%
\begin{Usage}
\begin{verbatim}
generate_branch_changes(tree, probe_node_matrix)
\end{verbatim}
\end{Usage}
%
\begin{Arguments}
\begin{ldescription}
\item[\code{tree}] ape::phylo tree

\item[\code{probe\_node\_matrix}] matrix storing the methylation status across all
probes and nodes on a tree
\end{ldescription}
\end{Arguments}
%
\begin{Value}
a structure containing a list of probes with differentially 
methylated probes for each node, a list of branch labels for the number of
total changes  in methylation, a list of branch labels for the number of 
gains in methylation, a list of branch labels for the number of losses in 
methylation
\end{Value}
%
\begin{Examples}
\begin{ExampleCode}
probe_information <- generate_branch_changes(tree, probe_node_matrix)
relevant_probes <- probe_information$relevant_probes
total_changes_label <- probe_information$total_changes_label
meth_gains_label <- probe_information$meth_gains_label
meth_losses_label <- probe_information$meth_losses_label
... some visualization 
\end{ExampleCode}
\end{Examples}
\inputencoding{utf8}
\HeaderA{generate\_probe\_node\_matrix}{Fully generate probe-node matrix with available data VALIDATED}{generate.Rul.probe.Rul.node.Rul.matrix}
%
\begin{Description}\relax
This function fully generates the probe-node matrix with a given 
tree
\end{Description}
%
\begin{Usage}
\begin{verbatim}
generate_probe_node_matrix(consensus_vector, tree)
\end{verbatim}
\end{Usage}
%
\begin{Arguments}
\begin{ldescription}
\item[\code{consensus\_vector}] consensus vector data frame for each cell in group 
names

\item[\code{tree}] ape::phylo tree
\end{ldescription}
\end{Arguments}
%
\begin{Value}
probe\_node\_matrix
\end{Value}
%
\begin{Examples}
\begin{ExampleCode}
tree_info <- generate_probe_node_matrix(consensus_vector, 
upgma_tree(consensus_vector))
tree <- tree_info[1]
probe_node_matrix <- tree_info[2]
... some visualization 
\end{ExampleCode}
\end{Examples}
\inputencoding{utf8}
\HeaderA{get\_leaves}{Leaves of a tree}{get.Rul.leaves}
%
\begin{Description}\relax
Given a tree, this function returns the leaves
\end{Description}
%
\begin{Usage}
\begin{verbatim}
get_leaves(tree)
\end{verbatim}
\end{Usage}
%
\begin{Arguments}
\begin{ldescription}
\item[\code{tree}] ape::phylo tree
\end{ldescription}
\end{Arguments}
%
\begin{Value}
leafs on a given tree
\end{Value}
%
\begin{Examples}
\begin{ExampleCode}
leaves <- get_leaves(tree)
... some visualization 
\end{ExampleCode}
\end{Examples}
\inputencoding{utf8}
\HeaderA{get\_probes\_of\_gene}{Probes along a given gene VALIDATED}{get.Rul.probes.Rul.of.Rul.gene}
%
\begin{Description}\relax
This function returns a list of probes along a particular gene
\end{Description}
%
\begin{Usage}
\begin{verbatim}
get_probes_of_gene(reference, gene)
\end{verbatim}
\end{Usage}
%
\begin{Arguments}
\begin{ldescription}
\item[\code{reference}] reference file

\item[\code{gene}] gene name
\end{ldescription}
\end{Arguments}
%
\begin{Value}
list of probes along a particular gene
\end{Value}
%
\begin{Examples}
\begin{ExampleCode}
gene_probes <- get_probes_of_gene(reference, gene)
... some visualization 
\end{ExampleCode}
\end{Examples}
\inputencoding{utf8}
\HeaderA{get\_reference}{Returns reference file VALIDATED}{get.Rul.reference}
%
\begin{Description}\relax
This function returns a reference file given location. Any preprocessing 
steps for the reference will also be added in this function.
\end{Description}
%
\begin{Usage}
\begin{verbatim}
get_reference(reference_file)
\end{verbatim}
\end{Usage}
%
\begin{Arguments}
\begin{ldescription}
\item[\code{reference\_file}] directory location of where given IDAT file is located
\end{ldescription}
\end{Arguments}
%
\begin{Value}
beta matrix without probes located on sex chromosomes.
\end{Value}
%
\begin{Examples}
\begin{ExampleCode}
reference <- get_reference('~/references/hg38/annotation/R/
EPIC.hg38.manifest.rds')
... some visualization 
\end{ExampleCode}
\end{Examples}
\inputencoding{utf8}
\HeaderA{get\_relevant\_genes}{Relevant genes for gain and loss of methylation VALIDATED}{get.Rul.relevant.Rul.genes}
%
\begin{Description}\relax
This function returns relevant gene lists for each node that exhibit a 
gain or loss in methylation status.
\end{Description}
%
\begin{Usage}
\begin{verbatim}
get_relevant_genes(tree, relevant_probes, reference)
\end{verbatim}
\end{Usage}
%
\begin{Arguments}
\begin{ldescription}
\item[\code{tree}] ape::phylo tree

\item[\code{relevant\_probes}] list of differentially methylated probes for each 
node

\item[\code{reference}] reference file
\end{ldescription}
\end{Arguments}
%
\begin{Value}
list of differentially methylated genes for each node
\end{Value}
%
\begin{Examples}
\begin{ExampleCode}
relevant_genes <- get_relevant_genes(tree, relevant_probes, reference)
... some visualization 
\end{ExampleCode}
\end{Examples}
\inputencoding{utf8}
\HeaderA{get\_root}{Root of a tree VALIDATED}{get.Rul.root}
%
\begin{Description}\relax
Returns the root of a tree structure.
\end{Description}
%
\begin{Usage}
\begin{verbatim}
get_root(tree)
\end{verbatim}
\end{Usage}
%
\begin{Arguments}
\begin{ldescription}
\item[\code{tree}] ape::phylo tree
\end{ldescription}
\end{Arguments}
%
\begin{Value}
node corresponding the the root of the tree
\end{Value}
%
\begin{Examples}
\begin{ExampleCode}
root <- get_root(tree)
... some visualization 
\end{ExampleCode}
\end{Examples}
\inputencoding{utf8}
\HeaderA{grow\_off\_leaf}{Grow a tree at a given leaf node}{grow.Rul.off.Rul.leaf}
%
\begin{Description}\relax
Add a pair of leaves branching off of the given leaf node for the tree. 
Asigns the two newly added leaves labels from node\_names the old leaf where 
the two branches were add as an internal node.
\end{Description}
%
\begin{Usage}
\begin{verbatim}
grow_off_leaf(tree, leaf, node_names)
\end{verbatim}
\end{Usage}
%
\begin{Arguments}
\begin{ldescription}
\item[\code{tree}] ape::phylo tree

\item[\code{leaf}] node number

\item[\code{node\_names}] list of two node names
\end{ldescription}
\end{Arguments}
%
\begin{Value}
new tree topology--two branches added to given leaf and labeled 
accordingly
\end{Value}
%
\begin{Examples}
\begin{ExampleCode}
tree <- grow_off_leaf(tree, 1, c('Fibroblasts', 'Fibrocytes'))
... some visualization 
\end{ExampleCode}
\end{Examples}
\inputencoding{utf8}
\HeaderA{inherit\_parental\_state}{Inherit parental states to ambiguous nodes VALIDATED}{inherit.Rul.parental.Rul.state}
%
\begin{Description}\relax
This function will recursively walk through a given tree and assigns 
parental states to ambiguous children nodes
\end{Description}
%
\begin{Usage}
\begin{verbatim}
inherit_parental_state(root, node_matrix, i)
\end{verbatim}
\end{Usage}
%
\begin{Arguments}
\begin{ldescription}
\item[\code{root}] root of non-proper subtree

\item[\code{node\_matrix}] matrix obtained from create\_node\_matrix function

\item[\code{i}] row in node matrix, start at 1
\end{ldescription}
\end{Arguments}
%
\begin{Value}
None
\end{Value}
%
\begin{Examples}
\begin{ExampleCode}
inherit_parental_state(get_root(tree), node_matrix, 1)
... some visualization 
\end{ExampleCode}
\end{Examples}
\inputencoding{utf8}
\HeaderA{initialize\_probe\_node\_matrix}{Initialize the probe-node matrix VALIDATED}{initialize.Rul.probe.Rul.node.Rul.matrix}
%
\begin{Description}\relax
This function simply initializes a probe-node matrix across given nodes and 
probes with zeros.
\end{Description}
%
\begin{Usage}
\begin{verbatim}
initialize_probe_node_matrix(consensus_vector, tree)
\end{verbatim}
\end{Usage}
%
\begin{Arguments}
\begin{ldescription}
\item[\code{consensus\_vector}] consensus vector data frame for each cell in group 
names

\item[\code{tree}] ape::phylo tree
\end{ldescription}
\end{Arguments}
%
\begin{Value}
initialized probe\_node\_matrix
\end{Value}
%
\begin{Examples}
\begin{ExampleCode}
probe_node_matrix <- initialize_probe_node_matrix(consensus_vector, 
upgma_tree(consensus_vector))
... some visualization 
\end{ExampleCode}
\end{Examples}
\inputencoding{utf8}
\HeaderA{ninternal\_nodes}{Number of internal nodes on a tree VALIDATED}{ninternal.Rul.nodes}
%
\begin{Description}\relax
Given a tree, this function returns the number of internal nodes on tree.
\end{Description}
%
\begin{Usage}
\begin{verbatim}
ninternal_nodes(tree)
\end{verbatim}
\end{Usage}
%
\begin{Arguments}
\begin{ldescription}
\item[\code{tree}] ape::phylo tree
\end{ldescription}
\end{Arguments}
%
\begin{Value}
number of internal nodes on a given tree
\end{Value}
%
\begin{Examples}
\begin{ExampleCode}
number_of_internal_nodes = ninternal_nodes(tree)
... some visualization 
\end{ExampleCode}
\end{Examples}
\inputencoding{utf8}
\HeaderA{nleaves}{Number of leaves of a tree}{nleaves}
%
\begin{Description}\relax
Given a tree, this function returns the number of leaves on a tree.
\end{Description}
%
\begin{Usage}
\begin{verbatim}
nleaves(tree)
\end{verbatim}
\end{Usage}
%
\begin{Arguments}
\begin{ldescription}
\item[\code{tree}] ape::phylo tree
\end{ldescription}
\end{Arguments}
%
\begin{Value}
number of leafs of a given tree
\end{Value}
%
\begin{Examples}
\begin{ExampleCode}
number_of_leaves = nleaves(tree)
... some visualization 
\end{ExampleCode}
\end{Examples}
\inputencoding{utf8}
\HeaderA{nnodes}{Total number of nodes on a tree VALIDATED}{nnodes}
%
\begin{Description}\relax
Given a tree, this function returns the number of nodes on a tree.
\end{Description}
%
\begin{Usage}
\begin{verbatim}
nnodes(tree)
\end{verbatim}
\end{Usage}
%
\begin{Arguments}
\begin{ldescription}
\item[\code{tree}] ape::phylo tree
\end{ldescription}
\end{Arguments}
%
\begin{Value}
number of nodes on a tree
\end{Value}
%
\begin{Examples}
\begin{ExampleCode}
number_of_nodes = nnodes(tree)
... some visualization 
\end{ExampleCode}
\end{Examples}
\inputencoding{utf8}
\HeaderA{nprobes}{Number of probes in data VALIDATED}{nprobes}
%
\begin{Description}\relax
This function returns the number of probes in the data
\end{Description}
%
\begin{Usage}
\begin{verbatim}
nprobes(consensus_vector)
\end{verbatim}
\end{Usage}
%
\begin{Arguments}
\begin{ldescription}
\item[\code{consensus\_vector}] consensus vector data frame for each cell in group 
names
\end{ldescription}
\end{Arguments}
%
\begin{Value}
number of probes
\end{Value}
%
\begin{Examples}
\begin{ExampleCode}
number_of_probes <- nprobes(consensus_vector)
... some visualization 
\end{ExampleCode}
\end{Examples}
\inputencoding{utf8}
\HeaderA{read\_IDAT}{Load beta data}{read.Rul.IDAT}
%
\begin{Description}\relax
Using sesame and sesame, this function loads beta data using IDAT files 
stored in a given directory
\end{Description}
%
\begin{Usage}
\begin{verbatim}
read_IDAT(IDAT_dir)
\end{verbatim}
\end{Usage}
%
\begin{Arguments}
\begin{ldescription}
\item[\code{IDAT\_dir}] directory location of where given IDAT file is located
\end{ldescription}
\end{Arguments}
%
\begin{Value}
beta matrix
\end{Value}
%
\begin{Examples}
\begin{ExampleCode}
betas <- read_IDAT('~/20191212_GEO_datasets/GSE110554')
... some visualization 
\end{ExampleCode}
\end{Examples}
\inputencoding{utf8}
\HeaderA{reconstruct\_internal\_nodes}{Reconstruct internal nodes VALIDATED}{reconstruct.Rul.internal.Rul.nodes}
%
\begin{Description}\relax
This function will recursively walk through a given tree and reconstruct 
all of its internal states using Fitch's algorithm
\end{Description}
%
\begin{Usage}
\begin{verbatim}
reconstruct_internal_nodes(root, node_matrix, i)
\end{verbatim}
\end{Usage}
%
\begin{Arguments}
\begin{ldescription}
\item[\code{root}] root of non-proper subtree

\item[\code{node\_matrix}] matrix obtained from create\_node\_matrix function

\item[\code{i}] row in node matrix, start at 1
\end{ldescription}
\end{Arguments}
%
\begin{Value}
None
\end{Value}
%
\begin{Examples}
\begin{ExampleCode}
reconstruct_internal_nodes(get_root(tree), node_matrix, 1)
... some visualization 
\end{ExampleCode}
\end{Examples}
\inputencoding{utf8}
\HeaderA{relabel\_node\_matrix}{Relabel node in probe-node matrix}{relabel.Rul.node.Rul.matrix}
%
\begin{Description}\relax
Relabel node label n\_old with node label n\_new in probe-node matrix.
\end{Description}
%
\begin{Usage}
\begin{verbatim}
relabel_node_matrix(probe_node_matrix, n_old, n_new)
\end{verbatim}
\end{Usage}
%
\begin{Arguments}
\begin{ldescription}
\item[\code{probe\_node\_matrix}] matrix storing the methylation status across all
probes and nodes on a tree

\item[\code{n\_old}] node number

\item[\code{n\_new}] node number
\end{ldescription}
\end{Arguments}
%
\begin{Value}
relabeled probe\_node\_matrix with node label n\_old with node label
n\_new
\end{Value}
%
\begin{Examples}
\begin{ExampleCode}
probe_node_matrix <- relabel_node_matrix(probe_node_matrix, 7, 10)
... some visualization 
\end{ExampleCode}
\end{Examples}
\inputencoding{utf8}
\HeaderA{relabel\_tree}{Relabel node on tree}{relabel.Rul.tree}
%
\begin{Description}\relax
Relabel node label n\_old with node label n\_new on tree.
\end{Description}
%
\begin{Usage}
\begin{verbatim}
relabel_tree(tree, n_old, n_new)
\end{verbatim}
\end{Usage}
%
\begin{Arguments}
\begin{ldescription}
\item[\code{tree}] ape::phylo tree

\item[\code{n\_old}] node number

\item[\code{n\_new}] node number
\end{ldescription}
\end{Arguments}
%
\begin{Value}
relabeled tree with node label n\_old with node label
n\_new
\end{Value}
%
\begin{Examples}
\begin{ExampleCode}
tree <- relabel_tree(tree, 7, 10)
... some visualization 
\end{ExampleCode}
\end{Examples}
\inputencoding{utf8}
\HeaderA{remove\_NA\_probes}{Removes NA probes VALIDATED}{remove.Rul.NA.Rul.probes}
%
\begin{Description}\relax
Selects and removes NA probes from consensus vectors
\end{Description}
%
\begin{Usage}
\begin{verbatim}
remove_NA_probes(consensus_vector)
\end{verbatim}
\end{Usage}
%
\begin{Arguments}
\begin{ldescription}
\item[\code{consensus\_vector}] consensus vector data frame for each cell in group 
names
\end{ldescription}
\end{Arguments}
%
\begin{Value}
consensus vector without NA probes
\end{Value}
%
\begin{Examples}
\begin{ExampleCode}
consensus_vector <- remove_NA_probes(consensus_vector)
... some visualization 
\end{ExampleCode}
\end{Examples}
\inputencoding{utf8}
\HeaderA{remove\_branches\_from\_node}{Cut tree at specified node}{remove.Rul.branches.Rul.from.Rul.node}
%
\begin{Description}\relax
Given a tree and a selected node, this function removes the branches 
stemming off of that node, thus making that node a leaf. It returns this 
altered tree.
\end{Description}
%
\begin{Usage}
\begin{verbatim}
remove_branches_from_node(tree, node)
\end{verbatim}
\end{Usage}
%
\begin{Arguments}
\begin{ldescription}
\item[\code{tree}] ape::phylo tree

\item[\code{node}] node number
\end{ldescription}
\end{Arguments}
%
\begin{Value}
new tree topology--node is now a leaf and all of node's descendants 
are removed.
\end{Value}
%
\begin{Examples}
\begin{ExampleCode}
tree <- remove_branches_from_node(tree, node)
... some visualization 
\end{ExampleCode}
\end{Examples}
\inputencoding{utf8}
\HeaderA{remove\_cell\_types}{Remove specific cell types from betas matrix VALIDATED}{remove.Rul.cell.Rul.types}
%
\begin{Description}\relax
This function removes specified group names. Common names include MIX or 
are typically heterogeneous mixtures of cells.
\end{Description}
%
\begin{Usage}
\begin{verbatim}
remove_cell_types(betas, remove_names)
\end{verbatim}
\end{Usage}
%
\begin{Arguments}
\begin{ldescription}
\item[\code{betas}] matrix (rows are cell names, columns are probe names)

\item[\code{remove\_names}] list of cell names in betas
\end{ldescription}
\end{Arguments}
%
\begin{Value}
beta matrix without specified cell names
\end{Value}
%
\begin{Examples}
\begin{ExampleCode}
betas <- remove_cell_types(betas, c('A.mix', 'B.mix', ...))
... some visualization 
\end{ExampleCode}
\end{Examples}
\inputencoding{utf8}
\HeaderA{remove\_intermediate\_probes}{Removes intermediate probes VALIDATED}{remove.Rul.intermediate.Rul.probes}
%
\begin{Description}\relax
Selects and removes intermediate probes from consensus vectors
\end{Description}
%
\begin{Usage}
\begin{verbatim}
remove_intermediate_probes(consensus_vector)
\end{verbatim}
\end{Usage}
%
\begin{Arguments}
\begin{ldescription}
\item[\code{consensus\_vector}] consensus vector data frame for each cell in group 
names
\end{ldescription}
\end{Arguments}
%
\begin{Value}
consensus vector without intermediate probes
\end{Value}
%
\begin{Examples}
\begin{ExampleCode}
consensus_vector <- remove_intermediate_probes(consensus_vector)
... some visualization 
\end{ExampleCode}
\end{Examples}
\inputencoding{utf8}
\HeaderA{remove\_sex\_chr}{Remove probes on sex chromosomes VALIDATED}{remove.Rul.sex.Rul.chr}
%
\begin{Description}\relax
This function removes probes located on sex chromosomes because these 
probes have a disproportionate global methylation status
\end{Description}
%
\begin{Usage}
\begin{verbatim}
remove_sex_chr(reference, betas)
\end{verbatim}
\end{Usage}
%
\begin{Arguments}
\begin{ldescription}
\item[\code{reference}] reference file

\item[\code{betas}] matrix (rows are probe names, columns are cell names)
\end{ldescription}
\end{Arguments}
%
\begin{Value}
beta matrix without probes located on sex chromosomes.
\end{Value}
%
\begin{Examples}
\begin{ExampleCode}
betas <- remove_sex_chr(reference, betas)
... some visualization 
\end{ExampleCode}
\end{Examples}
\inputencoding{utf8}
\HeaderA{remove\_unapplicable\_probes}{Removes all inapplicable probes VALIDATED}{remove.Rul.unapplicable.Rul.probes}
%
\begin{Description}\relax
Selects and removes NA, intermediate, and zero parsimony probes from 
consensus vectors
\end{Description}
%
\begin{Usage}
\begin{verbatim}
remove_unapplicable_probes(consensus_vector)
\end{verbatim}
\end{Usage}
%
\begin{Arguments}
\begin{ldescription}
\item[\code{consensus\_vector}] consensus vector data frame for each cell in group 
names
\end{ldescription}
\end{Arguments}
%
\begin{Value}
consensus vector without inapplicable probes
\end{Value}
%
\begin{Examples}
\begin{ExampleCode}
consensus_vector <- remove_unapplicable_probes(consensus_vector)
... some visualization 
\end{ExampleCode}
\end{Examples}
\inputencoding{utf8}
\HeaderA{remove\_zero\_parsimony\_probes}{Removes zero parsimony probes VALIDATED}{remove.Rul.zero.Rul.parsimony.Rul.probes}
%
\begin{Description}\relax
Selects and removes zero parsimony probes from consensus vectors
\end{Description}
%
\begin{Usage}
\begin{verbatim}
remove_zero_parsimony_probes(consensus_vector)
\end{verbatim}
\end{Usage}
%
\begin{Arguments}
\begin{ldescription}
\item[\code{consensus\_vector}] consensus vector data frame for each cell in group 
names
\end{ldescription}
\end{Arguments}
%
\begin{Value}
consensus vector without zero parsimony probes
\end{Value}
%
\begin{Examples}
\begin{ExampleCode}
consensus_vector <- remove_zero_parsimony_probes(consensus_vector)
... some visualization 
\end{ExampleCode}
\end{Examples}
\inputencoding{utf8}
\HeaderA{save\_data}{Saves specified data with variable name to working directory VALIDATED}{save.Rul.data}
%
\begin{Description}\relax
Given data, a variable name, and a location for that data to be stored, 
this function simply saves data.
\end{Description}
%
\begin{Usage}
\begin{verbatim}
save_data(data, variable_name, dir)
\end{verbatim}
\end{Usage}
%
\begin{Arguments}
\begin{ldescription}
\item[\code{data}] data structure

\item[\code{variable\_name}] name of variable/file

\item[\code{dir}] directory to which the data should be saved
\end{ldescription}
\end{Arguments}
%
\begin{Value}
None
\end{Value}
%
\begin{Examples}
\begin{ExampleCode}
save_data(betas, 'betas', getwd())
... some visualization 
\end{ExampleCode}
\end{Examples}
\inputencoding{utf8}
\HeaderA{show\_gene\_track}{Track the single probe resolution change of a gene}{show.Rul.gene.Rul.track}
%
\begin{Description}\relax
Displays constructed tree with node labels for the proportion of each probe 
state for a selected gene. This is saved at the given filename
\end{Description}
%
\begin{Usage}
\begin{verbatim}
show_gene_track(
  filename,
  tree,
  gene,
  thermo_prop_internal_nodes,
  thermo_prop_leaves
)
\end{verbatim}
\end{Usage}
%
\begin{Arguments}
\begin{ldescription}
\item[\code{filename}] filename to which the plot will be saved

\item[\code{tree}] ape::phylo tree

\item[\code{gene}] gene name

\item[\code{thermo\_prop\_leaves}] thermo proportion plot for leaves of all of
the probe states

\item[\code{thermo\_prop\_node}] thermo proportion plot for internal nodes of all of 
the probe states
\end{ldescription}
\end{Arguments}
%
\begin{Value}
None
\end{Value}
%
\begin{Examples}
\begin{ExampleCode}
show_gene_track('tree_CD81', tree, 'CD81', thermo_prop_internal_nodes, 
thermo_prop_leaves)
... some visualization 
\end{ExampleCode}
\end{Examples}
\inputencoding{utf8}
\HeaderA{show\_heatmap}{Relabel node in probe-node matrix}{show.Rul.heatmap}
%
\begin{Description}\relax
Relabel node label n\_old with node label n\_new in probe-node matrix.
\end{Description}
%
\begin{Usage}
\begin{verbatim}
show_heatmap()
\end{verbatim}
\end{Usage}
%
\begin{Arguments}
\begin{ldescription}
\item[\code{probe\_node\_matrix}] matrix storing the methylation status across all
probes and nodes on a tree

\item[\code{n\_old}] node number

\item[\code{n\_new}] node number
\end{ldescription}
\end{Arguments}
%
\begin{Value}
relabeled probe\_node\_matrix with node label n\_old with node label
n\_new
\end{Value}
%
\begin{Examples}
\begin{ExampleCode}
probe_node_matrix <- relabel_node_matrix(probe_node_matrix, 7, 10)
... some visualization 
\end{ExampleCode}
\end{Examples}
\inputencoding{utf8}
\HeaderA{show\_tree}{Display tree VALIDATED}{show.Rul.tree}
%
\begin{Description}\relax
Displays constructed tree and saves it at the given filename.
\end{Description}
%
\begin{Usage}
\begin{verbatim}
show_tree(filename, tree, title, edge_labels)
\end{verbatim}
\end{Usage}
%
\begin{Arguments}
\begin{ldescription}
\item[\code{filename}] filename to which the plot will be saved

\item[\code{tree}] ape::phylo tree

\item[\code{title}] title name

\item[\code{edge\_labels}] optional labels for each branch
\end{ldescription}
\end{Arguments}
%
\begin{Value}
None
\end{Value}
%
\begin{Examples}
\begin{ExampleCode}
show_tree('tree.pdf', tree, 'Hematopoietic Tree')
... some visualization 
\end{ExampleCode}
\end{Examples}
\inputencoding{utf8}
\HeaderA{track\_gene}{Proportion of probes methylated, unmethylated, ambiguous at each node  across a tree VALIDATED}{track.Rul.gene}
%
\begin{Description}\relax
This function returns the proportion of each probes that are methylated, 
unmethylated, ambiguous at each node across a tree.
\end{Description}
%
\begin{Usage}
\begin{verbatim}
track_gene(tree, probe_node_matrix, reference, gene)
\end{verbatim}
\end{Usage}
%
\begin{Arguments}
\begin{ldescription}
\item[\code{tree}] ape::phylo tree

\item[\code{probe\_node\_matrix}] matrix storing the methylation status across all
probes and nodes on a tree

\item[\code{reference}] reference file

\item[\code{gene}] gene name
\end{ldescription}
\end{Arguments}
%
\begin{Value}
a structure composed of two lists that contain methylation state 
proportions for internal nodes and methylation state proportions for leaves
\end{Value}
%
\begin{Examples}
\begin{ExampleCode}
gene_information <- track_gene(tree, probe_node_matrix, reference, 'CD81')
thermo_prop_internal_nodes = gene_information$thermo_prop_internal_nodes
thermo_prop_leaves = gene_information$thermo_prop_leaves
... some visualization 
\end{ExampleCode}
\end{Examples}
\inputencoding{utf8}
\HeaderA{upgma\_tree}{UPGMA tree construction VALIDATED}{upgma.Rul.tree}
%
\begin{Description}\relax
This function uses the UPGMA algorithm to construct a phylogenetic tree 
using methylation data
\end{Description}
%
\begin{Usage}
\begin{verbatim}
upgma_tree(consensus_vector)
\end{verbatim}
\end{Usage}
%
\begin{Arguments}
\begin{ldescription}
\item[\code{consensus\_vector}] consensus vector data frame for each cell in group 
names
\end{ldescription}
\end{Arguments}
%
\begin{Value}
ape::phylo tree constructed using UPGMA algorithm
\end{Value}
%
\begin{Examples}
\begin{ExampleCode}
tree <- upgma_tree(consensus_vector)
... some visualization 
\end{ExampleCode}
\end{Examples}
